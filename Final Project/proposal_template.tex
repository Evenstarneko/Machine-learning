\documentclass{article}
\usepackage[final]{proposal_style}
\usepackage[utf8]{inputenc} % allow utf-8 input
\usepackage[T1]{fontenc}    % use 8-bit T1 fonts
\usepackage{hyperref}       % hyperlinks
\usepackage{url}            % simple URL typesetting
\usepackage{booktabs}       % professional-quality tables
\usepackage{amsfonts}       % blackboard math symbols
\usepackage{nicefrac}       % compact symbols for 1/2, etc.
\usepackage{microtype}      % microtypography
\usepackage{graphicx}
\usepackage[many]{tcolorbox}
\usepackage{trimclip}
\usepackage{listings}
\usepackage{multicol}
\usepackage{environ}% http://ctan.org/pkg/environ
\usepackage{wasysym}
\usepackage{array}
\newcommand{\Checked}{{\LARGE \XBox}}%
\newcommand{\Unchecked}{{\LARGE \Square}}%

\pagenumbering{gobble}

\title{CS 475/675 Project Proposal}
% TODO replace with your project title

\author{
  Student 1, Student 2, Student 3, Student 4\\
  JHED 1, JHED 2, JHED 3, JHED 4
  % TODO replace with your names and JHEDs
}

\begin{document}
\maketitle

\begin{abstract}
The abstract should consist of two sentences describing the motivation for your project and your proposed methods.
% TODO write a two sentence abstract
\end{abstract}

\section{Project choice}

Choose either a {\bf methods} or {\bf applications} project, and a subarea from the below table.
\begin{table}[!h]
\centering
\def\arraystretch{2}
\begin{tabular}{c c c c c }
\toprule
% TODO replace "\Unchecked" with "\Checked" to choose an Applications project
\multicolumn{5}{l}{\Checked \bf Applications} \\
% TODO replace "\Unchecked" with "\Checked" on one of the five options below to choose that type of Applications project
\Unchecked Genomics data & 
\Unchecked Healthcare data & 
\Unchecked Text data &
\Checked Image data &
\Unchecked Finance data \\
\midrule
% TODO replace "\Unchecked" with "\Checked" to choose an Methods project
\multicolumn{5}{l}{\Unchecked \bf Methods} \\
% TODO replace "\Unchecked" with "\Checked" on one of the five options below to choose that type of Methods project
\Unchecked Fairness in ML &
\Unchecked Interpretable ML &
\Unchecked Graphical Models &
\Unchecked Robust ML &
\Unchecked Privacy in ML \\
\bottomrule
\end{tabular}
\end{table}

\section{Introduction}	
Face detection and recognition has been used nowadays in many areas, ranging from entertainment to security. In this project, we will develop a neural network that can make certain predictions given an image of human face. The input for the network will be face of a human. The single human face should be the dominant element in the image. The output will be predictions made to the person based on the image. Possible predictions include age, gender, and race.

\section{Dataset and Features}
We currently plan to use UTKFace dataset on the training of the network. A link to the dataset is included in reference part. Each entry of data consists of the following components: the image; four labels: age, gender, race, and date of collection of the image; landmarks on the face. We will not use the date as it is irrelavent to our task. Image are in RGB form with varying sizes. We will annotate the data by giving the position of the square where the face in teh image resides in. We will then crop and transform the image into a fixed size, with the face being the dominant element on the image. We will normalize the brightness of the image such that the impact from lighting in the surrounding environment will be eliminated.

\section{Methods}
Describe the methods you plan to use: what is your model's hypothesis class? your loss function? your optimization approach? Include enough information to demonstrate your understanding of the methods. You plan to use something not covered in class, explain it in 1-2 sentences, and provide a citation. 1-2 paragraphs.

\section{Deliverables}
These are ordered by how important they are to the project and how thoroughly you have thought them through. You should be confident that your ``must accomplish'' deliverables are achievable; one or two should be completed by the time you turn in your Nov 19 progress report.

\subsection{Must accomplish}

\begin{enumerate}
    \item A list of ~3 goals you must accomplish for a successful project
    \item etc.
\end{enumerate}

\subsection{Expect to accomplish}

\begin{enumerate}
    \item A list of ~3 goals you expect to accomplish as part of the project
    \item etc.
\end{enumerate}

\subsection{Would like to accomplish}

\begin{enumerate}
    \item A list of ~3 goals you hope to accomplish if everything goes well
    \item etc.
\end{enumerate}

\section*{References}
This section should include citations for: (1) Any papers on related work mentioned in the introduction.
(2) Papers describing methods that you used which were not covered in class.
(3) Code or libraries you downloaded and used.

\medskip
\small
% TODO replace these with your citations. These are just examples.
[1] Alexander, J.A.\ \& Mozer, M.C.\ (1995) Template-based algorithms
for connectionist rule extraction. In G.\ Tesauro, D.S.\ Touretzky and
T.K.\ Leen (eds.), {\it Advances in Neural Information Processing
  Systems 7}, pp.\ 609--616. Cambridge, MA: MIT Press.

[2] Bower, J.M.\ \& Beeman, D.\ (1995) {\it The Book of GENESIS:
  Exploring Realistic Neural Models with the GEneral NEural SImulation
  System.}  New York: TELOS/Springer--Verlag.

[3] UTKFace dataset https://susanqq.github.io/UTKFace/

\end{document}