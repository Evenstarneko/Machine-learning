\documentclass{article}
\usepackage[final]{proposal_style}
\usepackage[utf8]{inputenc} % allow utf-8 input
\usepackage[T1]{fontenc}    % use 8-bit T1 fonts
\usepackage{hyperref}       % hyperlinks
\usepackage{url}            % simple URL typesetting
\usepackage{booktabs}       % professional-quality tables
\usepackage{amsfonts}       % blackboard math symbols
\usepackage{nicefrac}       % compact symbols for 1/2, etc.
\usepackage{microtype}      % microtypography
\usepackage{graphicx}
\usepackage[many]{tcolorbox}
\usepackage{trimclip}
\usepackage{listings}
\usepackage{multicol}
\usepackage{environ}% http://ctan.org/pkg/environ
\usepackage{wasysym}
\usepackage{array}
\newcommand{\Checked}{{\LARGE \XBox}}%
\newcommand{\Unchecked}{{\LARGE \Square}}%

\pagenumbering{gobble}

\title{CS 475/675 Project Proposal}
% TODO replace with your project title

\author{
  Student 1, Student 2, Student 3, Student 4\\
  JHED 1, JHED 2, JHED 3, JHED 4
  % TODO replace with your names and JHEDs
}

\begin{document}
\maketitle

\begin{abstract}
The abstract should consist of two sentences describing the motivation for your project and your proposed methods.
% TODO write a two sentence abstract
\end{abstract}

\section{Project choice}

Choose either a {\bf methods} or {\bf applications} project, and a subarea from the below table.
\begin{table}[!h]
\centering
\def\arraystretch{2}
\begin{tabular}{c c c c c }
\toprule
% TODO replace "\Unchecked" with "\Checked" to choose an Applications project
\multicolumn{5}{l}{\Unchecked \bf Applications} \\
% TODO replace "\Unchecked" with "\Checked" on one of the five options below to choose that type of Applications project
\Unchecked Genomics data & 
\Unchecked Healthcare data & 
\Unchecked Text data &
\Unchecked Image data &
\Unchecked Finance data \\
\midrule
% TODO replace "\Unchecked" with "\Checked" to choose an Methods project
\multicolumn{5}{l}{\Unchecked \bf Methods} \\
% TODO replace "\Unchecked" with "\Checked" on one of the five options below to choose that type of Methods project
\Unchecked Fairness in ML &
\Unchecked Interpretable ML &
\Unchecked Graphical Models &
\Unchecked Robust ML &
\Unchecked Privacy in ML \\
\bottomrule
\end{tabular}
\end{table}

\section{Introduction}	
Explain the problem and why it is important. Discuss your motivation for pursuing this
problem. If necessary, give some background on published work in this related area. Clearly state what the input and output
is. Be very explicit: “The input to our algorithm is an {English sentence,
image, etc.}. We then use a {SVM, neural network, linear
regression, etc.} to predict {COVID case count, text sentiment, etc.}.”
This is very important since different teams have different inputs/outputs spanning different
application domains. Being explicit about this makes it easier for readers. 1-2 paragraphs.

\section{Dataset and Features}
Describe your dataset(s): how many training/validation/test examples do you have? What pre-processing did you do? What about normalization or data augmentation? What is the
resolution of your images? How is your time-series data discretized? Include a citation for the dataset(s) you are using.
You should also talk about the features you used. If you extracted
features using Fourier transforms, word2vec, PCA, etc. make sure to say so. If you have space, include one or two examples of your data in the report
(e.g. include an image, a slice of a time-series, etc.). 1-2 paragraphs.

\section{Methods}
Describe the methods you plan to use: what is your model's hypothesis class? your loss function? your optimization approach? Include enough information to demonstrate your understanding of the methods. You plan to use something not covered in class, explain it in 1-2 sentences, and provide a citation. 1-2 paragraphs.

\section{Deliverables}
These are ordered by how important they are to the project and how thoroughly you have thought them through. You should be confident that your ``must accomplish'' deliverables are achievable; one or two should be completed by the time you turn in your Nov 19 progress report.

\subsection{Must accomplish}

\begin{enumerate}
    \item A list of ~3 goals you must accomplish for a successful project
    \item etc.
\end{enumerate}

\subsection{Expect to accomplish}

\begin{enumerate}
    \item A list of ~3 goals you expect to accomplish as part of the project
    \item etc.
\end{enumerate}

\subsection{Would like to accomplish}

\begin{enumerate}
    \item A list of ~3 goals you hope to accomplish if everything goes well
    \item etc.
\end{enumerate}

\section*{References}
This section should include citations for: (1) Any papers on related work mentioned in the introduction.
(2) Papers describing methods that you used which were not covered in class.
(3) Code or libraries you downloaded and used.

\medskip
\small
% TODO replace these with your citations. These are just examples.
[1] Alexander, J.A.\ \& Mozer, M.C.\ (1995) Template-based algorithms
for connectionist rule extraction. In G.\ Tesauro, D.S.\ Touretzky and
T.K.\ Leen (eds.), {\it Advances in Neural Information Processing
  Systems 7}, pp.\ 609--616. Cambridge, MA: MIT Press.

[2] Bower, J.M.\ \& Beeman, D.\ (1995) {\it The Book of GENESIS:
  Exploring Realistic Neural Models with the GEneral NEural SImulation
  System.}  New York: TELOS/Springer--Verlag.

\end{document}